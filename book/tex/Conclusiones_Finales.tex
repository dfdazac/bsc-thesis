\chapter{Conclusiones Finales}

\section{Conclusiones}

A través del trabajo realizado, se analizaron y evaluaron diferentes métodos aplicables en la optimización del diseño y control de un motor eléctrico lineal de 100 W con una carga de 2 kg. Para el caso de la optimización del diseño, se seleccionaron y aplicaron tres procedimientos, y para el control del motor, se aplicó un método. Los resultados se validaron mediante simulación, cumpliéndose así los objetivos planteados inicialmente en el proyecto.

Una de las justificaciones planteadas indicaba el desarrollo de un producto que pudiera ser planteado como una alternativa para la producción de movimiento lineal, lo que conllevaba la obtención de un diseño en el que se buscara obtener valores de eficiencia dentro de aquellos obtenidos en trabajos relacionados, y el diseño de un sistema de control, de forma que se propusiera un sistema completo listo para ser construido e implementado.

El trabajo llevado a cabo demostró requerír de herramientas provenientes de distintas áreas con las cuales se tenía experiencia, como el electromagnetismo, la teoría de las máquinas eléctricas, la inteligencia computacional y el control. Por otro lado, fue necesario el estudio de áreas como la teoría de los motores lineales, el diseño de experimentos y la teoría de optimización. Esto es un indicador de cómo el diseño de un producto final requiere de una serie de conocimientos variados, que además pueden incluir otros no contemplados dentro de los alcances del proyecto, como la ciencia de materiales, la interferencia electromagnética, la construcción de máquinas eléctricas y los costos relacionados con esta tarea, entre otros.

En general, se observó la presencia de iteraciones en varios niveles, en diferentes etapas del trabajo, en las cuales era necesario repetir algún proceso con el fin de mejorar algún resultado (la eficiencia obtenida en el diseño inicial, el error producido por un metamodelo, el valor de un óptimo obtenido a través de un procedimiento de optimización, y las características dinámicas producidas por un controlador) a través de un cambio en el método original o la prueba de un método distinto (como la relajación de las restricciones en el tamaño del motor, la prueba con diferentes metamodelos y métodos de optimización, y la evaluación de diferentes tipos de controladores), en lo que puede verse como una generalización del NFLT, donde el objetivo de ``mejorar un resultado'' implica un proceso de optimización para el cual no existe a priori un método que garantice el mejor resultado. Una vez se ha estudiado el problema y se ha experimentado con el mismo, puede obtenerse un criterio que oriente decisiones futuras.

\section{Trabajos futuros}

Como trabajo futuro principal se propone la construcción del MLR y la implementación del controlador diseñado, con el fin de observar su comportamiento en la vida real y compararlo con las características predichas por las herramientas utilizadas en el proyecto. Aún cuando se intentó utilizar modelos fieles, como los estudiados con el FEM, que incluyen una definición precisa de la geometría del motor y la saturación en los materiales ferromagnéticos, estos siguen siendo modelos aproximados de la realidad, por lo que una implementación real serviría como método final de validación. 

Igualmente, como un trabajo posterior a la construcción e implementación del MLR y el controlador, se propone examinar sus resultados en términos del rizado en el empuje producido por el motor, la eficiencia total y los costos de implementación, con el fin de realimentar el proceso de diseño y producir un nuevo sistema con mejores especificaciones, haciendo uso del marco de trabajo planteado para este fin durante el desarrollo de este trabajo.

Finalmente, se propone la publicación de los resultados obtenidos durante las etapas de diseño, metamodelado, optimización y control, con el fin de realizar un aporte que pueda ser considerado en trabajos relacionados.