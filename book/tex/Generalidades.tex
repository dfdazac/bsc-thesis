\chapter{Generalidades}

\section{Planteamiento del problema}

% Optimización: ¿por qué? (en general)
La toma de decisiones es para las personas un proceso que hace necesaria, en general, la evaluación de un conjunto de posibles opciones y, por medio de uno o varios criterios, la selección de una en particular, a lo que sigue su ejecución, y por último, la realimentación de los resultados que se añade a la experiencia en el proceso, en el que el fin, casi siempre, consiste en la toma de la mejor decisión, partiendo de la existencia de lo que significa la ``mejor'' decisión. Este fin, en conjunto con lo que representan las actividades asociadas al mismo, se manifiesta en la teoría de la optimización.

Para un problema dado, en muchos casos es necesario hacer uso de soluciones que, no necesariamente siendo óptimas, cumplen con los requisitos para las cuales han sido creadas. La proposición de una solución inicial subóptima es incluso obligatoria en los casos en que no se conoce con anterioridad el problema, cuando las restricciones de tiempo no permiten un estudio más detallado, o cuando el problema es de fácil solución. Si, aún no siendo óptimas, estas soluciones cumplen sus objetivos a cabalidad, ¿por qué es necesario dedicar un esfuerzo adicional en la búsqueda de las soluciones óptimas y discriminarlas de las que no lo son? Probablemente, porque este tipo de soluciones no brindan garantías sobre características importantes como la eficiencia y el costo, entre otras. Una solución que cumple con el objetivo pero que utiliza grandes cantidades de recursos y es poco eficiente, podría incluso generar un problema mayor que aquel para el cual fue propuesta. Es por esto importante para la ingeniería contar con herramientas que hagan posible garantizar tanto la caracterización formal del concepto de la ``mejor'' solución y su evaluación, como la definición de un método de diseño que haga posible aproximarse a esta.

% En ingeniería: ¿por qué?
Asumiendo que se cuenta con una medida para evaluar el mejor diseño, un método que puede resultar acertado en algunas situaciones es el de la enumeración sucesiva, por medio del cual se evalúan todas las posibles combinaciones de las variables involucradas en el diseño. Sin embargo, cuando estas variables pueden tomar valores en intervalos continuos, la tarea se vuelve imposible debido a que el espacio de búsqueda es infinito, incluso si el número de variables es pequeño. Este método, sin embargo, puede usarse utilizando la observación como herramienta, en conjunto con la prueba y error, de manera que se reduzca el espacio de búsqueda de acuerdo a los resultados obtenidos con diseños particulares. Este procedimiento puede resultar práctico y exitoso en muchos casos y ha sido fundamental para el desarrollo de la humanidad, pero para problemas de carácter complejo, con grandes cantidades de variables y espacios de búsqueda continuos, sigue siendo inviable.

Por otro lado, el análisis aproximado del problema por medio de un estudio de causa-efecto puede realizarse con el fin de obtener modelos matemáticos simplificados que, junto con la experiencia, forman la base para un método empírico de diseño que es válido para el problema y otros muy similares al tratado. Este método es útil cuando se tiene un conocimiento extenso del problema y se desea resolver uno del mismo tipo pero con variaciones pequeñas, alrededor de las condiciones en las cuales el modelo del método empírico fue concebido, por lo cual no es apropiado para problemas desconocidos o en los que se busca obtener un diseño óptimo.

De acuerdo a lo anterior, se observa que a pesar de la utilidad de ciertos métodos de diseño en la satisfacción de objetivos, estos pueden ser insuficientes, en términos prácticos o cuando se requiere obtener un diseño óptimo. Una alternativa en estos casos es el uso de la optimización, que requiere del desarrollo de un modelo matemático del problema por medio de una definición clara de las variables, sus restricciones y una función de costo de estas. Es un método más formal y que requiere un estudio profundo del problema, en el sentido de la obtención de un modelo matemático válido, y conocimientos especializados, para el cual se han desarrollado un gran número de técnicas cuya aplicabilidad varía de acuerdo a la situación, y realizan la búsqueda de un punto óptimo haciendo uso de métodos de optimización determinísticos o estocásticos. En general, los métodos determinísticos asumen condiciones sobre la función objetivo en cuanto a su definición sobre un dominio y su diferenciabilidad, aunque sin embargo, en algunos problemas la función objetivo puede no cumplir las condiciones necesarias o incluso puede ser desconocida, por lo que estas dejan de ser aplicables. Es posible, en este tipo de problemas, hacer uso de métodos estocásticos, que puedan ser aplicables a una amplia variedad de situaciones, de forma que sean robustos. Aunque no es posible garantizar que el resultado corresponde al óptimo global, la combinación de técnicas aleatorias que proporcionan la exploración diversa del espacio de búsqueda junto con técnicas de explotación que producen mejoras a nivel local, producen a largo plazo una solución que es aceptable en términos de la definición del óptimo para el problema. No obstante, es posible que algunos algoritmos no logren escapar de óptimos locales, y tampoco existe una regla clara de cuál aplicar cuando no se tiene información \textit{a priori} sobre el problema.

% En máquinas eléctricas: ¿por qué? ¿cómo? Introducir máquinas lineales.
% Son máquinas inventadas hace mucho tiempo pero con la reducción del costo en la electrónica han podido implementarse en los últimos años como reemplazo a tecnología existentes, aunque el desarrollo no es tan activo como en motores convencionales. Qué técnicas no se han usado en el diseño de estas máquinas y pueden probarse para conocer si son útiles o no y su desempeño en comparación con las que si han sido usadas. Lo mismo con el control.
Muchos problemas en ingeniería pueden formularse como problemas de optimización. Uno de estos es el del diseño de máquinas eléctricas, utilizadas como motores o generadores en diversas aplicaciones. Mientras que los motores rotatorios han sido estudiados y utilizados por décadas, los motores lineales son otro tipo de máquina eléctrica que, aunque igualmente es una tecnología antigua, hasta los finales del siglo veinte empezó a surgir como una alternativa para proporcionar movimiento lineal, en la forma de trenes de levitación magnética y bandas transportadoras. Los motores lineales son una tecnología que brinda diversas ventajas en áreas de transporte y la industria, aumentando las velocidades de transporte y disminuyendo los desgastes producidos en sistemas convencionales.  Gracias a la reducción de los costos en los dispositivos necesarios para su construcción y funcionamiento y los avances en la microelectrónica y el control, esta tecnología ha sido objeto de interés tanto en la industria como en la academia, aunque definitivamente no con la misma atención que han recibido los motores rotatorios. Los diversos métodos de diseño contemplan técnicas como las mencionadas anteriormente, desde la empírica hasta métodos de optimización o metaheurísticas que tienen como objetivos, entre otros, incrementar la eficiencia y reducir algunos fenómenos no deseables en el motor lineal, por lo que el diseño de este tipo de máquinas es claramente un problema de optimización en el que los objetivos se orientan al manejo de la energía y los materiales. 
% Control de máquinas eléctricas
Debido a las característícas de velocidad y precisión que ofrecen, los motores lineales pueden superar las especificaciones de métodos de movimiento lineal convencionales, como el uso de motores rotativos, engranajes, correas y poleas, cuando son utilizados en conjunto con sistemas de control de posición y/o velocidad. Un motor lineal, como sistema dinámico, es un sistema no lineal cuyas propiedades de estabilidad dependen del tipo de diseño, sujeto a ruido y perturbaciones externas, por lo que existe una gran variedad de estrategias de control aplicables de acuerdo a su topología y a los requisitos del sistema controlado.
% PREGUNTA

Teniendo en cuenta la anterior discusión, a partir del diseño de un motor lineal como un problema de optimización que puede resolverse por medio de diferentes métodos, y la necesidad de un sistema de control que permita cumplir especificaciones de la dinámica del motor, ¿es posible realizar una exploración de métodos de optimización aplicados a la obtención del diseño de un motor lineal, así como de diferentes estrategias de control de velocidad para este tipo de motor?

\section{Objetivos}

\subsection{Objetivo General}
Analizar y evaluar diferentes métodos aplicables en la optimización del diseño y el control de un motor eléctrico lineal, con el fin de seleccionar un procedimiento en específico para el diseño de un motor lineal de 100W con una carga de 2kg y un controlador de velocidad para el mismo.

\subsection{Objetivos Específicos}
\begin{itemize}
\item Identificar y evaluar diferentes configuraciones de motores lineales, con el fin de escoger una en particular para el diseño, especificando un criterio de selección adecuado. Posteriormente, estudiar y caracterizar el problema de optimización del diseño para esta configuración.

\item Realizar un análisis de al menos dos métodos de optimización para el diseño que sean aplicables al problema, teniendo en cuenta el estado del arte y el conocimiento sobre el problema, con el fin de seleccionar uno en específico.

\item Implementar el método de optimización seleccionado y obtener un diseño de un motor lineal que entregue una potencia de 100W con una carga de mínimo 2kg, de forma que se cumplan los objetivos planteados en la caracterización del problema de optimización.

\item Analizar la aplicabilidad de al menos dos estrategias de control de velocidad para el motor lineal, teniendo en cuenta el estado del arte y el conocimiento sobre el problema, con el fin de seleccionar una en particular.

\item Diseñar un sistema de control para el motor lineal que permita mantener la velocidad constante frente a cambios en el sistema y perturbaciones externas.

\item Realizar una simulación del sistema diseñado, compuesto por el motor lineal, la carga y el controlador; caracterizarlo y concluir sobre su desempeño.

\end{itemize}

\section{Justificación}

\subsection{Justificación Académica}

El proyecto curricular de Ingeniería Electrónica determina dentro del contenido programático una serie de cursos que son considerados como requisitos para obtener el título de Ingeniero Electrónico. Entre estos se encuentran cursos sobre motores eléctricos, control e inteligencia computacional, que contienen temas que son pertinentes para el desarrollo del proyecto. De obtenerse un trabajo con un soporte teórico y práctico riguroso en términos de la teoría de optimización, el uso de energía y de los materiales, se podrán obtener resultados consistentes y coherentes con los campos de aplicación de la ingeniería, que específicamente en la eléctrica y electrónica, presentan en su estado del arte la utilización de métodos de optimización en el diseño de máquinas eléctricas y la implementación de diferentes estrategias de control para estas, en los que no siempre existe una regla definida para la selección de un método de solución en específico. Esto hace necesario el estudio del problema y el análisis de las diferentes formas de solución antes de escoger una forma de solución en particular.

Además de la aplicación de los temas tratados en los cursos mencionados, se busca generar una propuesta que, a partir de los resultados obtenidos, motive la investigación y el desarrollo de proyectos relacionados en los cuales el uso de motores lineales es una opción, como en la industria y el área interdisciplinaria del transporte, en la cual intervienen elementos de la ingenieria civil, ambiental, economía y otras.

\subsection{Justificación Socio-Económica}

La obtención de un diseño óptimo garantiza la obtención del mínimo o máximos posibles para un diseño que además, cumple con una funcionalidad para la cual inicialmente fue concebido. Estas garantías pueden verse traducidas en la reducción de costos, en términos de recursos humanos, económicos, energéticos, de tiempo y del uso de materiales, que en general están relacionados con el aumento de la productividad en los procesos, un mejor aprovechamiento de los recursos, un menor impacto en el medio ambiente y la posibilidad de mejorar la calidad de vida de las personas, por lo que el tiempo invertido en la búsqueda de un diseño óptimo de un motor lineal se justifica por estas razones. Por otro lado, el diseño obtenido es realmente relevante cuando se especifica una estrategia de control que permite cumplir requisitos de desempeño para un tipo de aplicación en específico, además de mejorar su desempeño en cuanto al rechazo de perturbaciones, estabilidad y  la inmunidad al ruido externo, lo que contribuye al impacto positivo del diseño óptimo y lo convierte en una propuesta a considerar en aplicaciones industriales.

\subsection{Justificación Personal}

Considero que el tema a desarrollar propone una aplicación de los conocimientos que he adquirido en mi formación, sobre todo en aquellos que he encontrado más enriquecedores de acuerdo a mis intereses. Desarrollar un trabajo en el cual el objetivo es el diseño de un sistema óptimo y su control, es motivador en cuanto a que requiere de la exploración de la teoría de un campo interesante y en desarollo en el que aún existen problemas abiertos, así como de la posibilidad de obtener un producto que pueda ser utilizado en la solución de diversos problemas, como el del transporte en las ciudades, el cual fue la base para la concepción de este proyecto. De este modo, considero el desarrollo del proyecto como un reto que vale la pena tomar que enriquecerá mi formación académica, y con un fin socialmente justificado.

\section{Alcances y Limitaciones}

\begin{itemize}
  \item La búsqueda de un diseño óptimo para el motor lineal parte de un diseño inicial para el cual se desean mejorar características de acuerdo a la formulación del problema de optimización. No se garantiza, sin embargo, que el motor lineal diseñado sea el óptimo para cualquier especificación de una máquina eléctrica que produzca movimiento lineal.
  \item El diseño del motor se define como la especificación de una geometría y parámetros de funcionamiento como son el voltaje y las corrientes nominales de alimentación. No serán tenidos en cuenta otros efectos, como aquellos debidos a la temperatura, materiales, interferencia electromagnética, ni restricciones en el diseño debidas a estándares para máquinas eléctricas.
  \item La implementación de un prototipo no se contempla, debido a las limitaciones de tiempo y costo que esto implicaría.
  \item En algunos casos la complejidad de la simulación en los procesos de diseño y validación del motor y el controlador puede resultar computacionalmente costosa, lo que eventualmente podría extender los tiempos de desarrollo del proyecto.
  \item Para la realización del proyecto se trabajará con licencias de software que estén disponibles en la universidad.
\end{itemize}