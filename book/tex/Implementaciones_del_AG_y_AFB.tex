\chapter{Implementaciones del AG y AFB}

A continuación se lista el código utilizado para implementar el algoritmo genético y el algoritmo de forrajeo de bacterias, utilizando el lenguaje de programación MATLAB.

\section{Algoritmo Genético}

\begin{scriptsize}
\begin{verbatim}
function best = runga(f, vars, popSize, nGen, spress, pcross, pmut, varargin)
%%RUNGA Ejecuta el Algoritmo Genético para maximización. Esta versión
%asume que f es una función objetivo vectorizada.

%% Preparar
% Número de variables
n = size(vars, 1);
% Determinar si se graficará curva de evolución
plotFlag = false;
arg = strcmpi(varargin, 'Plot');
if any(arg), plotFlag = true; end;

%% Crear estructuras de datos
% Población inicial
pop = repmat(vars(:,2) - vars(:,1), 1, popSize).*rand(n, popSize) + ...
    repmat(vars(:,1), 1, popSize);
% Población intermedia
ipop = zeros(n, popSize);
% Historia de aptitud
minH = zeros(1, nGen);
maxH = zeros(1, nGen);

%% Iniciar evolución
for i = 1:nGen
    %% Los individuos viven y son puestos a prueba
    % Vector de valores de la función objetivo
    objf = f(pop);
    % Vector de eficiencia
    fmax = max(objf);
    fmin = min(objf);
    minH(i) = fmin;
    maxH(i) = fmax;
    eff = (1-spress)*(objf-fmin)/max([fmax-fmin, eps]) + spress;
    % Selección por ruleta
    wheel = cumsum(eff);
    
    %% Los más aptos tienen más probabilidad de reproducirse
    % Generar conjunto de apareamiento
    for j = 1:popSize
        spin = max(wheel) * rand(1);        
        ipop(:,j) = pop(:,find(wheel >= spin, 1));
    end
    
    % Reproducir y cruzar
    for j = 1:2:popSize-1
        if rand(1) < pcross
            child = 0.5* ipop(:,j) + 0.5*ipop(:,j+1);
            pop(:,j) = child;
            pop(:,j+1) = child;
        else
            pop(:,j) = ipop(:,j);
            pop(:,j+1) = ipop(:,j+1);
        end
    end
    
    %% Algunos de los hijos sufren mutaciones
    for j = 1:popSize
        if rand(1) < pmut
            idx = randi(n);
            pop(idx, j) = (vars(idx,2) - vars(idx,1))*rand(1) + vars(idx,1);
        end
    end    
end

%% Seleccionar el mejor individuo
[~, idx] = max(objf);
best = pop(:, idx);

%% Graficar historia
if plotFlag
    figure; hold on;
    plot(minH);
    plot(maxH, 'r');
    legend('J_{min}', 'J_{max}', 'location', 'SouthEast');
    xlabel('Generaciones');
    ylabel('Función de aptitud');
    grid on;
end
end
\end{verbatim}
\end{scriptsize}

\section{Algoritmo de Forrajeo de Bacterias}
\begin{scriptsize}
\begin{verbatim}
function best = runbf(f, vars, params, varargin)
%%RUNBF Ejecutar forrajeo de bacterias para maximización. Esta versión 
%asume que f es una función objetivo vectorizada. Los
%parámetros se definen por el vector params de longitud 10, como sigue:
%   params(1): Tamaño de la población (debe ser impar)
%   params(2): Número de eventos de quemotaxis
%   params(3): Longitud de nado
%   params(4): Número de eventos de reproducción
%   params(5): Número de eventos de dispersión
%   params(6): Probabilidad de dispersión
%   params(7): Concentración de atractor/repelente
%   params(8): Ancho del atractor
%   params(9): Ancho del repelente
%   params(10): Longitud de movimiento
%   params(11): Presión selectiva
%Esta es una lista de parámetros libres orientada a permitir la experimentación. 
%Sin embargo, los siguientes parámetros pueden utilizarse inicialmente para realizar pruebas:
%   params = [50, 100, 4, 4, 2, 0.25, 0.1, 0.2, 10, 0.1, 0.01];

%% Preparar

% Número de dimensiones
n = size(vars, 1);
% Tamaño de la población
S = params(1);
% Número de eventos de quemotaxis
Nc = params(2);
% Longitud de nado
Ns = params(3);
% Número de eventos de reproducción
Nre = params(4);
% Número de eventos de dispersión
Ned = params(5);
% Probabilidad de dispersión
ped = params(6);
% Concentración de atractor/repelente
Sar = params(7);
% Ancho del atractor
Wa = params(8);
% Ancho del repelente
Wr = params(9);
% Longitud de movimiento
ru = params(10);
% Presión selectiva
spress = params(11);

% Determinar si se graficará curva de evolución
plotFlag = false;
arg = strcmpi(varargin, 'Plot');
if any(arg), plotFlag = true; end;

%% Crear estructuras de datos
% Posiciones de la población de bacterias
P = repmat(vars(:,2) - vars(:,1), 1, S).*rand(n, S) + repmat(vars(:,1), 1, S);
% Historia de la función de nutrientes
Jhmin = zeros(Ned*Nre*Nc, 1);
Jhmax = zeros(Ned*Nre*Nc, 1);
simStep = 1;

%% Iniciar simulación
% Iniciar ciclo de dispersión
for i = 1:Ned
    % Iniciar ciclo de reproducción
    for j = 1:Nre
        %% Iniciar ciclo de quemotaxis
        for k = 1:Nc
            % Calcular la concentración de nutrientes
            J = f(P);
            
            % Guardar historia
            Jhmin(simStep) = min(J);
            Jhmax(simStep) = max(J);
            simStep = simStep + 1;
            
            % Añadir efecto célula a célula
            J = J + cell2cell(J);                       
            % Ciclo de nado
            for m = 1:Ns
                % Guardar actual superficie de nutrientes
                Jp = J; 
                % Calcular la siguiente posición de cada bacteria (en una
                % matriz de población intermedia) utlizando un vector aleatorio
                Pi = P + ru * randuv(n, S);
                % Inicializar siguientes posiciones con la posición actual
                Pnext = P;
                % Vector lógico que contiene índices de las bacterias
                % cuyos movimientos no violan las restricciones después de
                % nadar
                valid = sum(bsxfun(@gt, Pi, vars(:,1)) & bsxfun(@lt, Pi, vars(:,2)), 1) == n;
                % Actualizar posiciones válidas
                Pnext(:,valid) = Pi(:,valid);
                % Calcular nueva superficie de nutrientes
                J = f(Pnext);
                J = J + cell2cell(J);
                % Sólo si para cada bacteria la función de nutrientes es
                % mejor, esta es movida
                valid = bsxfun(@gt, J, Jp);
                P(:,valid) = Pnext(:,valid);
            end
        end
        % Fin del ciclo de quemotaxis
        
        %% Reproducción
        % Passino sugiere el uso de la historia de la función de nutrientes como
        % medida de aptitud para la selección y reproducción. Aquí se propone
        % el uso de la medida de eficiencia (usada en el AG) por simplicidad.
        
        % Actualizar la superficie de nutrientes después de la quemotaxis
        J = f(P);
        Jmax = max(J);
        Jmin = min(J);
        eff = (1-spress)*(J - Jmin)/max([Jmax - Jmin, eps]) + spress;
        % Ordenar bacterias en orden descendiente de eficiencia
        [~, idxs] = sort(eff, 'descend');
        P(:,:) = P(:,idxs);
        % La mitad de la población menos apta muere y el resto se divide en dos
        % copias idénticas
        P(:,S/2+1:end) = P(:,1:S/2);
    end
    %% Eventos de dispersión
    for j = 1:S
        if rand < ped
            idx = randi(n);
            P(idx, j) = (vars(idx,2) - vars(idx,1))*rand + vars(idx,1);
        end
    end  
end
%% Seleccionar el mejor individuo
% Calcular la superficie de nutrientes final
J = f(P);
[~, idx] = max(J);
best = P(:,idx);

%% Graficar
if plotFlag
    figure; hold on;
    plot(Jhmin);
    plot(Jhmax, 'r');
    xlabel('Pasos');
    ylabel('Función de Nutrientes');
    legend('J_{min}', 'J_{max}', 'location', 'SouthEast');
    xlim([1, length(Jhmin)]);
    grid on;
end

%% Función de efecto célula a célula
function Jcc = cell2cell(X)
    % Preparar espacio en memoria
    Jcc = zeros(1, S);
    % Calcular la función para cada bacteria
    for b = 1:S
        % Matriz de suma de cuadrados de la diferencia
        Ds = sum((repmat(X(:,b), 1, S) - X).^2, 1);
        Jcc(b) = Sar * (sum(exp(-Wa * Ds), 2) - sum(exp(-Wr * Ds), 2));
    end
end

%% Generador de vectores unitarios aleatorios
function V = randuv(n, S)
% Genera S vectores unitarios aleatorios de n dimensiones.
% Basado en el generador de Maxim Vedenyov's.
% Ver http://www.mathworks.com/matlabcentral/fileexchange/25363-random-unit-vector-generator
    V = zeros(n, S);
    for ii = 1:S
        % Repetir si el vector es muy pequeño
        while 1
            % Usar una distribución normal para garantizar distribución del ángulo uniforme
            v = randn(n, 1);
            % Cuadrado de la norma
            vsnorm = v'*v;
            if vsnorm > 1e-10
                % Normalizar
                v = v/sqrt(vsnorm);
                break;
            end
        end
        V(:, ii) = v;
    end
end

end
\end{verbatim}
\end{scriptsize}